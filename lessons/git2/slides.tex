\documentclass{beamer}

\usepackage[utf8]{inputenc}
\usepackage[T1]{fontenc}
\usepackage{lmodern,tgadventor}
\usepackage{textcomp}
\usepackage{tikz}
\usetikzlibrary{backgrounds,decorations.pathreplacing,decorations.text,fadings,trees}

\useinnertheme{rectangles}
\usecolortheme[accent=blue]{solarized}
\setbeamercolor{description item}{fg=solarizedAccent}
\beamertemplatenavigationsymbolsempty
\setbeamertemplate{footline}[frame number]

% Shell highlighting
\newcommand{\command}[1]{\textcolor{solarizedAccent}{\$}
                         \textcolor{solarizedRebase2}{#1}}
\newcommand{\comment}[1]{\textit{\# #1}}
\newcommand{\hiRed}[1]{\textcolor{solarizedRed}{#1}}
\newcommand{\hiGreen}[1]{\textcolor{solarizedGreen}{#1}}
\newcommand{\hiBlue}[1]{\textcolor{solarizedBlue}{#1}}
\newcommand{\hiYellow}[1]{\textcolor{solarizedYellow}{#1}}

% DAG commands
\newcommand\commit[2]{\node[commit] (#1) {};%
                      \node[clabel] (#1 label) at (#1) {\texttt{#1}: #2};}
\newcommand\ghost[1]{\coordinate (#1);}
\newcommand\connect[2]{\path[connections] (#1) to[out=90,in=-90] (#2);}
\newcommand\gittag[1]{%
 \begin{tikzpicture}[baseline]
  \node[fill=solarizedRebase02, rounded corners, inner sep=1pt, anchor=base] {
    \color{solarizedYellow}\vphantom{|}#1
   };
 \end{tikzpicture}%
}
\newcommand\branch[1]{%
 \begin{tikzpicture}[baseline]
  \node[fill=solarizedRebase02, rounded corners, inner sep=1pt, anchor=base] {
    \color{solarizedRed}\vphantom{|}#1
   };
 \end{tikzpicture}%
}

% http://tex.stackexchange.com/questions/55806/tikzpicture-in-beamer/55827#55827
\tikzset{
  invisible/.style={opacity=0},
  visible on/.style={alt=#1{}{invisible}},
  alt/.code args={<#1>#2#3}{%
    \alt<#1>{\pgfkeysalso{#2}}{\pgfkeysalso{#3}}
  },
}

\author{U of T Scientific Coders}
\title{An Intermediate Look at Git + GitHub}
\institute{University of Toronto}
\date{October 1, 2015}
\titlegraphic{%
 \begin{tikzpicture}[
    commit/.style={draw, circle, fill=solarizedViolet, inner sep=0pt,
                   minimum size=5pt},
    clabel/.style={right, outer sep=1em},
    connections/.style={draw, solarizedViolet}
  ]
  \useasboundingbox (0,1em) rectangle (10.7,6.5em); % Mysteriously chosen to fit...
  \matrix [column sep={1em,between origins}, row sep=\lineskip,
           ampersand replacement=\&]%
  {
   \commit{fd7a4e4}{\branch{gh-pages} Create 2015-10-08-Coworking4.markd\dots} \& \& \\
   \commit{c3cb768}{Merge pull request \#41 from mbonsma/gh-pages} \& \& \\
   \& \commit{e2764b7}{Incorporated PR comments into Biopython/less\dots} \& \\
   \& \commit{d212bdd}{Merge remote-tracking branch `upstream/gh-p\dots} \& \\
   \commit{85791b7}{Added start and end time to event post} \& \& \\
   \& \commit{b83b3e0}{Merge remote-tracking branch `upstream/gh-p\dots} \& \\
   \commit{0b7d78c}{Create 2015-10-01-MoreObGit.markdown} \& \& \\
   \commit{c188bb0}{Merge pull request \#48 from lwjohnst86/gh-pages} \& \& \\
   \ghost{ghost00} \& \ghost{ghost01} \& \\
  };
  % Left branch
  \connect{c3cb768}{fd7a4e4};
  \connect{85791b7}{c3cb768};
  \connect{0b7d78c}{85791b7};
  \connect{c188bb0}{0b7d78c};
  \connect{ghost00}{c188bb0};
  % Upper right branch
  \connect{e2764b7}{c3cb768};
  \connect{d212bdd}{e2764b7};
  \connect{85791b7}{d212bdd};
  % Lower right branch
  \connect{b83b3e0}{d212bdd};
  \connect{0b7d78c}{b83b3e0};
  \connect{ghost01}{b83b3e0};
  \fill [color=solarizedRebase03, path fading=north] (-1,-1.5) rectangle (16,0);
 \end{tikzpicture}%
}

\AtBeginSection[]{\frame{\sectionpage}}

\begin{document}

\begin{frame}[plain,noframenumbering]
 \titlepage
\end{frame}

\begin{frame}
 \frametitle{Outline}
 \tableofcontents
\end{frame}

\section{Review}

\begin{frame}
 \frametitle{Review}

 \begin{itemize}
  \item Configure your git client (\texttt{git config user.name} + \texttt{user.email})
  \item Create a git repository (\texttt{git init})
  \pause
  \begin{tikzpicture}[%
    remember picture,
    show background rectangle,
    background rectangle/.style={draw, visible on=<2->},
    visible on=<2->
   ]
   \begin{scope}[
     every node/.style={anchor=west, minimum height=1em,
                        font={\tiny\ttfamily}, inner sep=2.5pt},
     every edge/.style={thick},
     root/.style={},
     grow via three points={one child at (0.25,-0.75em) and
                            two children at (0.25,-0.75em) and (0.25,-1.25em)},
     edge from parent path={(\tikzparentnode.south) |- (\tikzchildnode.west)}
    ]
    \node[anchor=west, font={\scriptsize}, align=center]
     at (0,0) (work) {Working\\Directory};
    \node [root, anchor=north] at (work.south) (work root) {my\_project/}
     child { node (git folder) {.git/} }
     child { node (first folder) {foo/} }
     child { node {bar/} }
     child { node {baz.txt} }
     child { node {qux.txt} }
     child { node (last folder) {\dots} };
   \end{scope}
   \node[anchor=north west, font={\scriptsize}, xshift=4em]
    at (work.north east) (repository) {Repository};
   \begin{scope}[
     every node/.style={anchor=west, minimum height=1em,
                        font={\tiny\ttfamily}, inner sep=2.5pt},
     every edge/.style={thick},
     root/.style={},
     grow via three points={one child at (0.25,-0.75em) and
                            two children at (0.25,-0.75em) and (0.25,-1.25em)},
     edge from parent path={(\tikzparentnode.south) |- (\tikzchildnode.west)}
    ]
    \node[anchor=north, font={\scriptsize}, align=center]
     at (repository.south) (index) {Index/\\Staging Area};
    \node<3->[root, anchor=north, align=center]
     at (index.south) (snapshot) {\color{solarizedAccent} (Snapshot)}
     child { node {foo/} }
     child { node {baz.txt} }
     child { node {\dots} };
   \end{scope}
   \begin{scope}[
     commit/.style={draw, rectangle, rounded corners, fill=solarizedRebase02,
                    inner sep=1pt, minimum height=0.75em, minimum width=1em,
                    font={\tiny\sffamily}},
     connections/.style={draw}
    ]
    \node[anchor=west, font={\scriptsize}, align=center, xshift=1em] at (index.east)
     (history) {History};
    \matrix at (history.south)
      [column sep={1em,between origins}, row sep={1em,between origins},
       ampersand replacement=\&, yshift=-2.5em]%
    {
     \node[commit,visible on=<4->] (10) {\color{solarizedAccent} 10}; \\
     \node[commit] (9) {9}; \\
     \node[commit] (8) {8}; \\
     \ghost{7}; \\
    };
    \path[connections, visible on=<4->] (9) -- (10);
    \connect{8}{9};
    \path[connections, densely dotted] (7) -- (8);
   \end{scope}
   \begin{scope}[
     path/.style={->},
     label/.style={decorate, decoration={text along path, text align=center,
                                         text color=solarizedRebase0,
                                         raise=1pt}}
    ]
    \draw[path] (git folder.east) to[out=0,in=180] (repository.west);
    \draw (index.north west) rectangle (7.south -| history.east);
    \draw<3->[decorate, decoration={brace, raise=0.75em, aspect=0.45}]
      (first folder.east) |- (last folder.east) node[near start] (mid selection) {};
    \draw<3->[path] (mid selection.east) + (0.6em,0) to[out=0,in=180] (snapshot.west);
    \draw<4->[path] (snapshot.east) to[out=0,in=180] (10.west);
   \end{scope}
  \end{tikzpicture}
  \pause
  \item Start tracking a file with git (\texttt{git add})
  \pause
  \item Commit changes to the history (\texttt{git commit})
  \pause
  \item Check what's going on (\texttt{git status})
  \item Compare a file with the one in the history (\texttt{git diff})
  \item Look into your history (\texttt{git log})
 \end{itemize}
\end{frame}

\section{Viewing History}

\begin{frame}
 \frametitle{Viewing History}

 Viewing the log allows you to ``see'' history:
 \begin{itemize}
  \item \texttt{git log}
  \pause
  \item \texttt{git log \alert{<start>..<end>}}
  \item \texttt{git log \alert{-{}- <file>}}
  \item \texttt{git log \alert{-{}-oneline}}
  \item \texttt{git log \alert{-{}-graph}}
  \item \texttt{git log -{}-graph \alert{-{}-decorate}} \\
 \end{itemize}
 \pause
 \begin{itemize}
  \item \texttt{git blame <file>}
 \end{itemize}
 \pause
 \begin{itemize}
  \item \texttt{gitk} + \texttt{gitg} + other viewers
 \end{itemize}
\end{frame}

\section{Branching}

\begin{frame}
 \frametitle{Branches}

 What are branches?
 \begin{itemize}[<+->]
  \item Divergent commits (two commits with the same parent) could be
        considered ``virtual'' branches
  \item Branches are simply a \alert{named pointer} to a commit
  \item Branches automatically \emph{move forward} as commits are made
 \end{itemize}
 \pause[\thebeamerpauses]
 Why use them?
 \pause
 \begin{itemize}[<+->]
  \item They're cheap! Just pointers. No heavy changes, e.g., an extra
        directory in svn.
  \item To keep experimental work apart
  \item To separate trials
  \item To ease collaboration
 \end{itemize}
\end{frame}

\begin{frame}
 \frametitle{Branches}

 Managing branches:
 \begin{itemize}
  \item \texttt{git branch <name> [commit]}
  \item \texttt{git branch -d <name>}
  \item \texttt{git branch [-l]}
 \end{itemize}

 \pause
 Switching branches:
 \begin{itemize}
  \item \texttt{git checkout <branch name>}
  \item \texttt{git checkout -b <branch name> [commit]} --- Create and switch
        in one go
 \end{itemize}

 \pause
 Merging branches:
 \begin{itemize}
  \item \texttt{git merge <other branch name>}
  \item \texttt{git merge \alert{-{}-ff-only} <other branch name>}
  \item \texttt{git merge \alert{-{}-no-ff} <other branch name>}
 \end{itemize}
\end{frame}

\section{Collaborating with Others}

\begin{frame}
 \frametitle{Clones and Remotes}

 Clones are complete copies of a repository's history (i.e., excluding the
 index and working directory)
 \pause
 \begin{itemize}
  \item \texttt{git clone <URI>}
 \end{itemize}

 \pause
 Remotes are \emph{pointers} to other clones
 \pause
 \begin{itemize}
  \item \texttt{git remote [-v]}
  \item \texttt{git remote add <name> <URI>}
  \item \texttt{git remote rm <name>}
  \item Local branches can \emph{track} remote branches\\
        \texttt{git branch -u <remote branch> <local branch>}
 \end{itemize}

 \pause
 \emph{You} are responsible for syncing
 \begin{itemize}
  \item \texttt{git push [<remote>] [<branch>]}
  \item \texttt{git fetch [<remote>]}
  \item \texttt{git pull [<remote>]} --- \texttt{fetch} + \texttt{merge}
 \end{itemize}
\end{frame}

\begin{frame}[plain]
 \vfill
 \begin{center}
  \LARGE \color{solarizedAccent} GitHub Example
 \end{center}
 \vfill
\end{frame}

\part{Advanced Topics}

\begin{frame}[plain]
 \vfill
 \begin{center}
  \LARGE \color{solarizedAccent} Advanced Topics
 \end{center}
 \vfill
\end{frame}

\end{document}
